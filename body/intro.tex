% !Mode:: "TeX:UTF-8"

\chapter{引言}

    本模板根据天津大学硕士论文模板改编,适用于中山大学工学院硕士论文。

\section{使用方法}

    Windows 环境下winedt, pdftexify 编译,测试~ textlive 可用,TexStudio 可用,未测试~MacOS 环境。文件夹,文件名建议采用英文,不要空格。所有文件以~UTF-8格式保存。clean 用于清理文件编译过程中的缓存。
    
    更改页眉标题,

\section{例子}

\subsection{图表}

\begin{table}[htbp]
    \linespread{1.3}
    \caption{和~XX 的比较}
    \label{tab}
    \centering
    \begin{threeparttable}
    \small
\begin{tabular}
{p{0.76in}p{2.2in}p{1.5in}p{1.278in}}
\hline
    & a
    & b
    & 本文  \\
\hline
      A
    & Aa
    & Ab
    & c\\
%\hline
      线性化
    &\checkmark
    &$\times$
    &$\times$\\
\hline
\end{tabular}
\end{threeparttable}
\end{table}

\begin{figure}[htbp]
  \centering
  \includegraphics[width=2in]{sysu}\\
  \caption{技术路线图}\label{fig}
\end{figure}

\subsection{数学环境}


\subsubsection{公式}

\begin{align}
\label{eq}
    a+b=c
\end{align}

    引用公式~\eqref{eq}

\subsubsection{假设~Assumption}

%\begin{assumption}:\label{ass}
%    假设
%\end{assumption}

\subsubsection{定义~Definition}

\begin{definition}:\label{def}
    定义
\end{definition}

\subsubsection{命题~Proposition}

\begin{proposition}: \label{pro}
    命题
\end{proposition}

\subsubsection{定理~Theorem}

\begin{theorem}: \label{th}
    定理
\end{theorem}

\subsubsection{引理~Lemma}

\begin{lemma}: \label{le}
     引理
\end{lemma}




\subsubsection{推论~Corollary}

\begin{corollary}: \label{co}
     推论
\end{corollary}


\subsubsection{证明~Proof}

\begin{proof}
    证明
\end{proof}


\subsubsection{注~Remark}

\begin{remark}:

\end{remark}

\subsubsection{算法}


\subsection{引用}

    参考文献上标\cite{friesz2011,DH2017},使用~UTF-8 编码的~bib 文件,可引用中文参考文献,文献标识不能为中文。
