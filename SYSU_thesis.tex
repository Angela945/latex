% !Mode:: "TeX:UTF-8"

\def\usewhat{pdflatex}      % 定义编译方式 dvipdfmx 或者 pdflatex,默认为 dvipdfmx
                            % 方式编译,如果需要修改,只需改变花括号中的内容即可。

\documentclass[12pt,openright,twoside]{book}          % book 作为文档类
                                                    % 插入空白页可以设置openright

\input{setup/package}            % 定义本文所使用宏包

\graphicspath{{figures/}}       % 定义所有的.eps/.pdf 图片文件在figures 子目录下


\begin{document}


\begin{CJK*}{UTF8}{song}    % 开始中文字体使用

\input{setup/format}        % 完成对论文各个部分格式的设置

% ——————————————————————————————————————————————
% 以下是论文导言部分,包括论文的封面,中英文摘要和中文目录

\frontmatter
\fancypagestyle{plain}{
\fancyhf{}
\renewcommand{\headrulewidth}{0 pt}
\fancyfoot[C]{\song\xiaowu~\thepage~}
}

%%%%%%%%%%   封面   %%%%%%%%%%
\input{preface/cover}

%%%%%%%%%%   目录   %%%%%%%%%%
\defaultfont
%\clearpage{\pagestyle{empty}\cleardoublepage}
\clearpage
%\pagestyle{empty}
%\setcounter{page}{1}                                 % 单独从 1 开始编页码
%\pagenumbering{arabic}
\titleformat{\chapter}{\centering\sanhao\hei}{\chaptername}{2em}{} % 设置目录两字的格式

%\pdfbookmark[0]{目~~录}{mulu}
\addcontentsline{toc}{chapter}{目录}
%\titleformat{\chapter}{\centering\sihao\hei}\CJKnumber{\chaptername}{2em}{}
%\renewcommand{\chaptername}{\prechaptername\CJKnumber{\thechapter}\postchaptername}
\tableofcontents                                     % 中文目录

%\fancypagestyle{plain}{
%\fancyhf{}
%\renewcommand{\headrulewidth}{0 pt}
%\fancyfoot[C]{\song\xiaowu~\thepage~}
%}
\thispagestyle{plain}

% ——————————————————————————————————————————————
% 以下是论文正文,内容在body 子文件夹下

\mainmatter\defaultfont\sloppy\raggedbottom

\makeatletter
	\fancypagestyle{plain}{                              % 设置开章页眉页脚风格
		\fancyhf{}
		\fancyhead[OC]{\song\wuhao \@cheading}            % 首页页眉格式
        \fancyhead[EC]{\song\wuhao \@ctitle}          % 首页页眉格式
		\fancyfoot[C]{\song\xiaowu ~\thepage~}           % 首页页脚格式
		\renewcommand{\headrulewidth}{0.5pt}
		\renewcommand{\footrulewidth}{0pt}}
\makeatother

% 单独从 1 开始编页码
\setcounter{page}{1}
% chapter标题格式:小二,黑体,居中
%\titleformat{\chapter}{\centering\xiaoer\hei}{{\chaptername}}{2em}{}
%%%%%%%%%%   正文   %%%%%%%%%%
\floatname{algorithm}{算法}

% !Mode:: "TeX:UTF-8"

\chapter{引言}

    本模板根据天津大学硕士论文模板改编,适用于中山大学工学院硕士论文。

\section{使用方法}

    Windows 环境下winedt, pdftexify 编译,测试~ textlive 可用,TexStudio 可用,未测试~MacOS 环境。文件夹,文件名建议采用英文,不要空格。所有文件以~UTF-8格式保存。clean 用于清理文件编译过程中的缓存。
    
    更改页眉标题,

\section{例子}

\subsection{图表}

\begin{table}[htbp]
    \linespread{1.3}
    \caption{和~XX 的比较}
    \label{tab}
    \centering
    \begin{threeparttable}
    \small
\begin{tabular}
{p{0.76in}p{2.2in}p{1.5in}p{1.278in}}
\hline
    & a
    & b
    & 本文  \\
\hline
      A
    & Aa
    & Ab
    & c\\
%\hline
      线性化
    &\checkmark
    &$\times$
    &$\times$\\
\hline
\end{tabular}
\end{threeparttable}
\end{table}

\begin{figure}[htbp]
  \centering
  \includegraphics[width=2in]{sysu}\\
  \caption{技术路线图}\label{fig}
\end{figure}

\subsection{数学环境}


\subsubsection{公式}

\begin{align}
\label{eq}
    a+b=c
\end{align}

    引用公式~\eqref{eq}

\subsubsection{假设~Assumption}

%\begin{assumption}:\label{ass}
%    假设
%\end{assumption}

\subsubsection{定义~Definition}

\begin{definition}:\label{def}
    定义
\end{definition}

\subsubsection{命题~Proposition}

\begin{proposition}: \label{pro}
    命题
\end{proposition}

\subsubsection{定理~Theorem}

\begin{theorem}: \label{th}
    定理
\end{theorem}

\subsubsection{引理~Lemma}

\begin{lemma}: \label{le}
     引理
\end{lemma}




\subsubsection{推论~Corollary}

\begin{corollary}: \label{co}
     推论
\end{corollary}


\subsubsection{证明~Proof}

\begin{proof}
    证明
\end{proof}


\subsubsection{注~Remark}

\begin{remark}:

\end{remark}

\subsubsection{算法}


\subsection{引用}

    参考文献上标\cite{friesz2011,DH2017},使用~UTF-8 编码的~bib 文件,可引用中文参考文献,文献标识不能为中文。


%\include{body/pre}
%\include{body/mfd_system}
%\include{body/mfd_oc}
%\include{body/concl}

%%%%%%%%%%  参考文献  %%%%%%%%%%
\lhead{}
\rhead{}
\chead{\song\wuhao 中山大学硕士学位论文} % 覆盖设置页眉内容
\defaultfont
\bibliographystyle{references/TJUThesis}
%\bibliographystyle{GBT7714-2005NLang}
\phantomsection
\markboth{参考文献}{参考文献}
\addcontentsline{toc}{chapter}{参考文献}       % 参考文献加入到中文目录
%\nocite{*}                                     % 若将此命令屏蔽掉,则未引用的文献不会出现在文后的参考文献中
\bibliography{references/thesis}


%\titleformat{\chapter}{\centering\sihao\hei}\CJKnumber{\chaptername}{2em}{} % 重新设置想要的chapter格式

%% 如果使用第“一”章
\renewcommand{\chaptername}{\prechaptername\CJKnumber{\thechapter}\postchaptername}
%% 使用第“1”章
%%\renewcommand{\chaptername}{\prechaptername~\thechapter~\postchaptername}

% !Mode:: "TeX:UTF-8"

\markboth{附\quad 录}{附\quad 录}
\addcontentsline{toc}{chapter}{附\quad 录} % 添加到目录中

%\renewcommand\thechapter{\Alph{chapter}}
\renewcommand\thesection{\Alph{section}}
%\renewcommand\thesubsection{\thesection.\arabic{subsection}}
\renewcommand\theequation{\Alph{section}.\arabic{equation}}
%\renewcommand\theequation{\Alph{section}.\arabic{subsection}.\arabic{equation}}
\appendix

\chapter*{附\quad 录}\label{sec:num_sim}
%\lstset{numbers=none}


\section{附录~A}

    附录~A

\subsection{附录~A.1}\label{A}

    附录~A.1


% !Mode:: "TeX:UTF-8"

\markboth{缩略语与符号}{缩略语与符号}
\addcontentsline{toc}{chapter}{缩略语与符号} % 添加到目录中

\chapter*{缩略语与符号}

\begin{longtable}
{p{0.5in}p{2.8in}p{2in}}
    SYSU  & Sun Yat-sen University & 中山大学\\
\end{longtable}


\include{body/paper}
% !Mode:: "TeX:UTF-8"

\markboth{致\quad 谢}{致\quad 谢}
\addcontentsline{toc}{chapter}{致\quad 谢} % 添加到目录中

\chapter*{致\quad 谢}

    感谢~CCTV
    
% 如果需要加名字和日期(日期根据生成文档日期变更)
\begin{flushright}
  \begin{tabular}{cl}
    王五四 & \\
    二零一八年\CJKnumber{\the\month}月\CJKnumber{\the\day}日 &
  \end{tabular}
\end{flushright}


\clearpage
\end{CJK*}% 结束中文字体使用

\end{document}
